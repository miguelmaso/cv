\documentclass[12pt]{article}

\usepackage{cv}

\addbibresource{cv.bib}

\author{Miguel Masó, PhD}

\begin{document}
\maketitle
\email{miguel.maso@socotec.com} \qquad +34 717 712 217 \\
City: Barcelona (08006) \qquad
Birth date: 9/4/1989 \qquad
ORCID: 0000-0003-0962-8550


\section{Employment history}
\begin{tabularx}{\linewidth}{lX}
    2022 - \enskip \dots &
    \textbf{SOCOTEC Spain} - Buildings and cities \newline
    Head of development and innovation \\

    2022 - \enskip \dots &
    \textbf{Universitat Politècnica de Catalunya (UPC)} - Barcelona School of Civil Engineering \newline
    Adjunct lecturer \\ 

    2017 - 2022 &
    \textbf{International Centre for Numerical Methods in Engineering (CIMNE)} \newline
    Doctoral student with FPI grant \\
\end{tabularx}


\section{Education}
\begin{tabularx}{\linewidth}{lX}
    2017 - 2022 &
    \textbf{PhD in Civil Engineering} \newline
    UPC, Barcelona, Spain \\

    2013 - 2015 &
    \textbf{Bachelor's degree in Philosophy and Theology} \newline
    Ponitificia Universtià della Santa Croce, Rome, Italy \\

    2007 - 2013 &
    \textbf{Ingeniero de Caminos, Canales y Puertos} (1st and 2nd cycle studies) \newline
    UPC, Barcelona, Spain \\
\end{tabularx}


\section{Languages}
Native Spanish - Native Catalan - C1 English - B1 Italian


\section{CV Summary}
After studying civil engineering and a period studying philosophy and theology in Rome, he began
a doctorate in civil engineering at CIMNE. During the initial years of his doctoral studies, he combined
research with work for social purposes at the Montseny Foundation.

Durign the research period at CIMNE, he worked on the stabilization of finite elements for the hyperbolic
differential equations, with focus on numerical implementation, multi-physics coupling, and multi-scale
resolution. These developments have been succesfully applied to various projects. In addition to
the finite element stabilization, he also researched on the particle finite element method, either
with moving mesh and fixed mesh.

Currently, he is working at the Building and Cities department of Socotec-Spain. He has participated
in several international projects for structural design, taking on responsibilities of vibration
assessment and plasticity. He is leading the innovation and development department.


\section{Publications}
\nocite{*}
\printbibliography[heading={subbibliography}, title={Journal papers}, type=article]
\printbibliography[heading={subbibliography}, title={Theses}, type=thesis]
\printbibliography[heading={subbibliography}, title={Conferences}, type=inproceedings]


\section{Stays and other activities}
\textbf{Guest. Centro de Investigación en Matemáticas.} Purpose: Research. Comparable tasks: WP3 of project
TCAiNMaND-PIRSES-GA-2013-612607 Numerical Methods for Real Time Computations Research on stabilized
formulations the Finite Element Method applied to hyperbolic conservation laws. 18/07/2017 - 19/10/2017.


\section{Teaching activities}
\begin{tabularx}{\linewidth}{lX}
    2022 - \enskip \dots &
    \textbf{Structural Dynamics.} Credits: 7'5. Degree: Master's degree in Structural \& Construction Engineering. UPC. \\

    2022 - 2022 &
    \textbf{Structural Analysis.} Credits: 7'5. Degree: Master's degree in Civil Engineering. UPC. \\

    2017 - 2017 &
    \textbf{Numerical Methods.} Credits: 6. Degree: Bachelor's degree in Materials Engineering. UPC. \\
\end{tabularx}


\section{Miscellaneous}
\begin{itemize}
    \itemsep=-.3em
    \item \textbf{Skills:} Computational Mechanics, Fire Engineering, Ocean Engineering, Vibration Assessment
    \item \textbf{Programming:} C++, Python, C\#, .NET
    \item \textbf{Tools:} Robot Structural Analysis API, Revit API
    \item \textbf{Miscellaneous:} ENVISION SP 2023-2024
\end{itemize}


\end{document}
