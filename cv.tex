\documentclass[12pt]{article}

\usepackage[english]{babel}
\usepackage{cv}

\addbibresource{cv.bib}

\author{Miguel Masó, PhD}

\begin{document}
\maketitle
\email{miguel.maso@socotec.com} \qquad +34 717 712 217 \\
City: Barcelona (08006) \qquad
Birth date: 9/4/1989 \qquad
ORCID: 0000-0003-0962-8550


\section{Employment history}
\begin{tabularx}{\linewidth}{lX}
    2025 - today &
    \textbf{International Centre for Numerical Methods in Engineering (CIMNE)} \newline
    Postdoctoral researcher \\

    2022 - today &
    \textbf{Universitat Politècnica de Catalunya (UPC)} - Barcelona School of Civil Engineering \newline
    Adjunct lecturer \\ 

    2022 - 2025 &
    \textbf{SOCOTEC Spain} - Buildings and cities \newline
    Specialist Engineer, Head of development and innovation \\

    2017 - 2022 &
    \textbf{International Centre for Numerical Methods in Engineering (CIMNE)} \newline
    Doctoral student with FPI grant \\
\end{tabularx}


\section{Education}
\begin{tabularx}{\linewidth}{lX}
    2017 - 2022 &
    \textbf{PhD in Civil Engineering} \newline
    UPC, Barcelona, Spain \\

    2013 - 2015 &
    \textbf{Bachelor's degree in Philosophy and Theology} \newline
    Ponitificia Universtià della Santa Croce, Rome, Italy \\

    2007 - 2013 &
    \textbf{Ingeniero de Caminos, Canales y Puertos} (1st and 2nd cycle studies) \newline
    UPC, Barcelona, Spain \\
\end{tabularx}


\section{Languages}
Native Spanish - Native Catalan - C1 English - B1 Italian


\section{CV Summary}
After studying civil engineering and a period studying philosophy and theology in Rome, he began
a doctorate in civil engineering at CIMNE. During the initial years of his doctoral studies, he combined
research with work for social purposes at the Montseny Foundation.

Durign the research period at CIMNE, he worked on the stabilization of finite elements for the hyperbolic
differential equations, with focus on numerical implementation, multi-physics coupling, and multi-scale
resolution. These developments have been succesfully applied to various projects. In addition to
the finite element stabilization, he also researched on the particle finite element method, either
with moving mesh and fixed mesh.

After a three-year stint in the structural engineering consultancy at SOCOTEC, he has returned
to academia as a postdoctoral researcher.


\section{Publications}
\nocite{*}
\printbibliography[heading={subbibliography}, title={Journal papers}, type=article]
\printbibliography[heading={subbibliography}, title={Theses}, type=thesis]
\printbibliography[heading={subbibliography}, title={Conferences}, type=inproceedings]


\section{Stays and other activities}
\textbf{Guest. Centro de Investigación en Matemáticas.} Purpose: Research. Comparable tasks: WP3 of project
TCAiNMaND-PIRSES-GA-2013-612607 Numerical Methods for Real Time Computations Research on stabilized
formulations the Finite Element Method applied to hyperbolic conservation laws. 18/07/2017 - 19/10/2017.


\section{Teaching activities}
\begin{tabularx}{\linewidth}{lX}
    2022 - today &
    \textbf{Structural Dynamics.} Credits: 7'5. Degree: Master's degree in Structural \& Construction Engineering. UPC. \\

    2022 - 2022 &
    \textbf{Structural Analysis.} Credits: 7'5. Degree: Master's degree in Civil Engineering. UPC. \\

    2017 - 2017 &
    \textbf{Numerical Methods.} Credits: 6. Degree: Bachelor's degree in Materials Engineering. UPC. \\
\end{tabularx}


\printbibliography[heading={bibintoc}, title={Participation in R+D+i projects}, type=project]


\section{Divulgation activities}
\begin{tabularx}{\linewidth}{l}
    \parbox{\linewidth}{
        Subject: Sostenibilidad de las estructuras. Criterios para la descarbonización. \\
        Promoting entity: Asociación de Consultores de Estructuras (ACE) \\
        Organizing entity: Fundació Privada Institut d'Estudis Estructurals \\
        Dates: 16/10/2024 - 13/11/2024
    }
\end{tabularx}


\section{Participation in industry projects}
\begin{tabularx}{\linewidth}{X}
    New Velindre Cancer Centre, Cardiff (2022-2025) SACYR \newline
    Structural design;
    Robustness assessment against disproportional collapse;
    Vibration assessment for isolation of medical equipment. \\
    
    Shushah Island Resort, Stage 3 (2022-2024) RBTA \newline
    Structural design. \\

    Étude de mode de ruine de six bâtiments industriels (2023) SOCOTEC-France \newline
    CFD analysis of fire;
    Structural analysis of thermo-plastic deformations. \\

    Anàlisi de patologia de la cúpula de la sala capitular del monestir de Pedralbes (2023) ICUB \newline
    Forensic analysis of crack propagation. \\

    Mercury Tower, Malta (2024) Zaha Hadid \newline
    Vibration assessment on serviceability limits for a flying theatre. \\

    Proposta correctiva a la cúpula de la sala capitular del monestir de Pedralbes (2024) ICUB \newline
    Structural design. \\

    Hotel Puerto Málaga (2024) b720 \newline
    Structural analysis;
    Deep foundation design. \\

    Bridgs on new Genome Campus, Hinxton (2025) Evolve \newline
    Assessment on structural design. \\

    Residencial Torre Garellano (2025) IDOM \newline
    CFD Analysis of wind loads on facade. \\

    BIM developments internal project (2023-2025) SOCOTEC-Spain \newline
    Interconnectivity and automation processes of BIM modelling from calculation models. \\

    BIM developments (2025) Hormipresa \newline
    Development of an automated process of revision of assemblies lifting.
\end{tabularx}


\section{Miscellaneous}
\begin{itemize}
    \itemsep=-.3em
    \item \textbf{Skills:} Computational Mechanics, Fire Engineering, Ocean Engineering, Vibration Assessment
    \item \textbf{Programming:} C++, Python, Julia, C\#, .NET, Git
    \item \textbf{Tools:} Robot Structural Analysis API, Revit API
    \item \textbf{Miscellaneous:} ENVISION SP 2023-2024
\end{itemize}


\end{document}
