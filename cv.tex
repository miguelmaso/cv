\documentclass[12pt]{article}

\usepackage[english]{babel}
\usepackage{cv}

\addbibresource{cv.bib}

\author{Miguel Masó, PhD}

\begin{document}
\maketitle
\begin{tabularx}{\linewidth}{lCr}
\email{miguel.maso@upc.edu} &
+34 717 712 217 &
Chartered Civil Engineer, Nº 36285\\[-.2em]
City: Barcelona (08006) &
Birth date: 9/4/1989 &
ORCID: 0000-0003-0962-8550    
\end{tabularx}


\section{Employment history}
\begin{tabularx}{\linewidth}{lX}
    2025 - \enspace \dots &
    \textbf{International Centre for Numerical Methods in Engineering (CIMNE)} \newline
    Postdoctoral researcher \\

    2022 - \enspace \dots &
    \textbf{Universitat Politècnica de Catalunya (UPC)} - Barcelona School of Civil Engineering \newline
    Adjunct lecturer \\ 

    2022 - 2025 &
    \textbf{SOCOTEC Spain} - Buildings and cities \newline
    Specialist Engineer, Head of Development and Innovation \\

    2017 - 2022 &
    \textbf{International Centre for Numerical Methods in Engineering (CIMNE)} \newline
    PhD Candidate (FPI Fellowship) \\
\end{tabularx}


\section{Education}
\begin{tabularx}{\linewidth}{lX}
    2017 - 2022 &
    \textbf{PhD in Civil Engineering} \newline
    UPC, Barcelona, Spain \\

    2013 - 2015 &
    \textbf{Bachelor's degree in Philosophy and Theology} \newline
    Pontificia Universtià della Santa Croce, Rome, Italy \\

    2007 - 2013 &
    \textbf{Ingeniero de Caminos, Canales y Puertos} (equivalent to MSc in Civil Engineering) \newline
    UPC, Barcelona, Spain \\
\end{tabularx}


\section{Languages}
Spanish (native) - Catalan (Native) - English (C1) - Italian (B1)


\section{CV Summary}
Perfil Profesional Ingeniero de Caminos y Doctor en Mecánica Computacional con una sólida trayectoria híbrida que combina la ingeniería estructural avanzada y el desarrollo de software.
Experto en cerrar la brecha entre la teoría numérica y la aplicación industrial, con dominio de C++, Python, .NET y APIs de BIM.

Como Responsable de Desarrollo e Innovación en SOCOTEC, lideré la automatización de procesos de diseño y la interconectividad entre modelos de cálculo y BIM.
Mi formación doctoral en métodos numéricos (FEM, interacción fluido-estructura) me permite abordar problemas complejos de simulación física desde el código fuente hasta la solución práctica.
Actualmente investigador postdoctoral en CIMNE desarrollando formulaciones para nuevos materiales (polímeros electroactivos), busco aplicar mi capacidad técnica en entornos de desarrollo de software de ingeniería de alto nivel.

% After he graduated in civil engineering and a period studying philosophy and theology in Rome, he began
% a doctorate in civil engineering at CIMNE. During the initial years of his doctoral studies, he combined
% research with work for social purposes at the Montseny Foundation.

% During his research period at CIMNE, he worked on the stabilization of finite elements for hyperbolic
% differential equations, focusing on numerical implementation, multi-physics coupling, and multi-scale
% resolution. These developments have been successfully applied to various projects.

% Working in structural engineering consultancy at SOCOTEC, he was involved in RDI projects, the Wind\&Fire
% department, and the analysis of complex structures, focusing on dynamic effects. He is now working at CIMNE
% as a postdoctoral researcher, developing formulations for Electro-Active-Polymers (EAPs).


\section{Publications}
\nocite{*}
\printbibliography[heading={subbibliography}, title={Journal papers}, type=article]
\printbibliography[heading={subbibliography}, title={Theses}, type=thesis]
\printbibliography[heading={subbibliography}, title={Conferences}, type=inproceedings]


\section{Stays and other activities}
\textbf{Visiting researcher, Centro de Investigación en Matemáticas.} Purpose: Research. Comparable tasks: WP3 of project
TCAiNMaND-PIRSES-GA-2013-612607 Numerical Methods for Real Time Computations Research on stabilized
formulations the Finite Element Method applied to hyperbolic conservation laws. 18/07/2017 - 19/10/2017.


\section{Teaching activities}
\begin{tabularx}{\linewidth}{lX}
    2022 - \enspace \dots &
    \textbf{Structural Dynamics.} Credits: 7'5. Degree: Master's degree in Structural \& Construction Engineering. UPC. \\

    2022 - 2022 &
    \textbf{Structural Analysis.} Credits: 7'5. Degree: Master's degree in Civil Engineering. UPC. \\

    2017 - 2017 &
    \textbf{Numerical Methods.} Credits: 6. Degree: Bachelor's degree in Materials Engineering. UPC. \\
\end{tabularx}


\printbibliography[heading={bibintoc}, title={Participation in R+D+i projects}, type=project]


\section{Divulgation activities}
\parbox{\linewidth}{
    Subject: Sostenibilidad de las estructuras. Criterios para la descarbonización. \\
    Promoting entity: Asociación de Consultores de Estructuras (ACE) \\
    Organizing entity: Fundació Privada Institut d'Estudis Estructurals \\
    Dates: 16/10/2024 - 13/11/2024
}


\section{Participation in industry projects}

\subsection{Advanced Structural Analysis \& Complex Dynamics}

\parbox{\linewidth}{
    Mercury Tower, Malta (2024) Zaha Hadid \newline
    \emph{Reto}: Evaluación de dinámica estructural para una atracción de "teatro volador" (Flying Theatre) en un rascacielos de diseño singular. \newline
    \emph{Solución}: Realicé el análisis avanzado de vibraciones para garantizar el cumplimiento de estrictos límites de servicio (SLS) y confort humano bajo cargas dinámicas.
}

\parbox{\linewidth}{
    New Velindre Cancer Centre, Cardiff (2022-2025) SACYR \newline
    \emph{Retos}: Diseño de infraestructura crítica sanitaria con requisitos estrictos de aislamiento y seguridad ante accidentes. \newline
    \emph{Soluciones}: Lideré los estudios de vibraciones para el aislamiento de equipos médicos de alta sensibilidad (MRI/Escáneres) asegurando la operatividad del centro y de robustez frente al colapso progresivo.
}

\subsection{Multiphysics ans forensic analysis}

\parbox{\linewidth}{
    Investigación Forense y Multifísica (Varios Proyectos) \newline
    Análisis Termo-mecánico (2023) SOCOTEC Francia: Ejecución de simulaciones acopladas CFD-Estructural para predecir modos de ruina en naves industriales bajo cargas de fuego, modelando deformaciones termo-plásticas. \newline
    Monasterio de Pedralbes (2024) Patrimonio: Análisis forense de patologías mediante modelado numérico de la propagación de grietas y diseño de propuestas correctivas para la estabilización de la cúpula histórica (S. XV). \newline
    Residencial Torre Garellano (2025) Arup: Análisis CFD para la determinación de cargas de viento en fachada en edificación de gran altura.
}

\subsection{Computational Design \& BIM Automation (R\&D)}

\parbox{\linewidth}{
    Head of Development \& Innovation (2022-2025) SOCOTEC Spain - Buildings \& cities \newline
    Automatización: Desarrollo de algoritmos propios (C\# / .NET / Python) para la interoperabilidad directa entre modelos de cálculo y software BIM (Revit/Robot), eliminando el retrabajo manual.
}

\parbox{\linewidth}{
    Consultor externo (2024 - 2025) Hormipresa \newline
    Control de Calidad: Implementación de procesos automatizados para la revisión de conjuntos de izado en prefabricados, optimizando la seguridad y reduciendo tiempos de revisión técnica.
}

% \parbox{\linewidth}{
%     New Velindre Cancer Centre, Cardiff (2022-2025) SACYR \newline
%     Structural design;
%     Robustness assessment against disproportional collapse;
%     Vibration assessment for isolation of medical equipment.}

% \parbox{\linewidth}{
%     Shushah Island Resort, Stage 3 (2022-2024) RBTA \newline
%     Structural design.}

% \parbox{\linewidth}{
%     Étude de mode de ruine de six bâtiments industriels (2023) SOCOTEC-France \newline
%     CFD analysis of fire;
%     Structural analysis of thermo-plastic deformations.}

% \parbox{\linewidth}{
%     Anàlisi de patologia de la cúpula de la sala capitular del monestir de Pedralbes (2023) ICUB \newline
%     Forensic analysis of crack propagation.}

% \parbox{\linewidth}{
%     Mercury Tower, Malta (2024) Zaha Hadid \newline
%     Vibration assessment on serviceability limits for a flying theatre.}

% \parbox{\linewidth}{
%     Proposta correctiva a la cúpula de la sala capitular del monestir de Pedralbes (2024) ICUB \newline
%     Structural design.}

% \parbox{\linewidth}{
%     Hotel Puerto Málaga (2024) b720 \newline
%     Structural analysis;
%     Deep foundation design.}

% \parbox{\linewidth}{
%     Bridges on new Genome Campus, Hinxton (2025) Evolve \newline
%     Revision and assessment of structural design.}

% \parbox{\linewidth}{
%     Residencial Torre Garellano (2025) Arup \newline
%     CFD Analysis of wind loads on facade.}

% \parbox{\linewidth}{
%     BIM developments internal project (2023-2025) SOCOTEC-Spain \newline
%     Interconnectivity and automation processes of BIM modelling from calculation models.}

% \parbox{\linewidth}{
%     BIM developments (2025) Hormipresa \newline
%     Development of an automated process for reviewing lifting assemblies.}


\section{Miscellaneous}
\begin{itemize}
    \itemsep=-.3em
    \item \textbf{Skills:} Computational Mechanics, Fire Engineering, Ocean Engineering, Vibration Assessment
    \item \textbf{Programming:} C++, Python, Julia, C\#, .NET, Git
    \item \textbf{Tools:} Robot Structural Analysis API, Revit API
    \item \textbf{Miscellaneous:} ENVISION Sustainability Professional 2023-2024
\end{itemize}


\end{document}
