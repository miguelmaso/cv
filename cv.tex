\documentclass[12pt]{article}

\usepackage[english]{babel}
\usepackage{cv}

\usepackage{fontawesome5}

\addbibresource{cv.bib}

\author{Miguel Masó, PhD}

\begin{document}
\maketitle
\begin{tabularx}{\linewidth}{lCr}
\email{miguel.maso@upc.edu} &
+34 717 712 217 &
Chartered Civil Engineer, Nº 36285\\[-.2em]
City: Barcelona (08006) &
Birth date: 9/4/1989 &
\faLinkedin \ \href{https://www.linkedin.com/in/miguel-maso/}{linkedin.com/in/miguel-maso} \\[-.2em]   
\end{tabularx}


\section{CV Summary}
Civil Engineer and PhD in Computational Mechanics with a hybrid background combining advanced structural engineering and software development. Experienced in bridging the gap between numerical theory and industrial application, with proficiency in C++, Python, .NET, and BIM APIs.

As Head of Development and Innovation at SOCOTEC, I spearheaded the automation of design processes and interoperability between calculation models and BIM. My doctoral background in numerical methods (FEM, Fluid-Structure Interaction) enables me to tackle complex physical simulation problems—from source code to practical solutions.

Currently as Postdoctoral Researcher at CIMNE developing formulations for new materials (electroactive polymers), I am seeking to apply my technical expertise within high-level engineering software development environments.


\section{Core competencies \& technical skills}
\begin{description}
    % \itemsep=-.3em
    \item[Advanced simulation \& physics]
        \emph{Computational mechanics}: Expertise in FEM, CFD, Particle-methods, Stabilization of hyperbolic equations.
        \emph{Structural analysis}: Vibration assessment, Disproportional collapse, CFD for Fire/Wind engineering, non-linear structural analysis.
        \emph{Ocean \& coastal}: Shallow water equations modelling, landslide-generated waves simulation.
    \item[Scientific programming \& automation]
        \emph{Languages}: C++ (high-performance solvers), Python \& Julia (Data processing/Scripting), C\# / .NET (Windows ecosystem).
        \emph{Plattforms}: Windows, Linux.
        \emph{BIM development}: Advanced interoperability using Revit API and Robot Structural Analysis API.
        \emph{Version control}: Git, CI/CD workflows.
    \item[Languages] Spanish (native) - Catalan (Native) - English (C1) - Italian (B1)
\end{description}


\section{Employment history}
\begin{tabularx}{\linewidth}{lX}
    2025 - \enspace \dots &
    \textbf{International Centre for Numerical Methods in Engineering (CIMNE)} \newline
    Postdoctoral researcher \\

    2022 - \enspace \dots &
    \textbf{Universitat Politècnica de Catalunya (UPC)} - School of Civil Engineering \newline
    Adjunct lecturer \\ 

    2022 - 2025 &
    \textbf{SOCOTEC Spain} - Buildings and cities \newline
    Specialist Engineer, Head of Development and Innovation \\

    2017 - 2022 &
    \textbf{International Centre for Numerical Methods in Engineering (CIMNE)} \newline
    PhD Candidate (FPI Fellowship) \\
\end{tabularx}


\section{Selected industry projects}

\subsection{Advanced Structural Analysis \& Complex Dynamics}
\parbox{\linewidth}{
    Mercury Tower, Malta (2024) Zaha Hadid \newline
    \emph{Reto}: Structural dynamic assessment for a "Flying Theatre" attraction within a signature high-rise building. \newline
    \emph{Solución}: Conducted advanced vibration analysis to ensure strict compliance with Serviceability Limit States (SLS) and human comfort criteria under dynamic loading scenarios.}

\parbox{\linewidth}{
    New Velindre Cancer Centre, Cardiff (2022-2025) SACYR \newline
    \emph{Retos}: Design of critical healthcare infrastructure with stringent requirements for isolation and safety against accidental loads. \newline
    \emph{Soluciones}: Led vibration studies for the isolation of highly sensitive medical equipment (MRI/Scanners). Led the analysis of structural robustness against progressive collapse ensuring facility operability.}

\subsection{Multiphysics and forensic analysis}
\parbox{\linewidth}{
    Thermo-mechanical analysis (2023) SOCOTEC Francia \newline
    Executed coupled CFD-Structural simulations to predict failure modes in industrial warehouses under fire loads, accurately modeling thermo-plastic deformations.}

\parbox{\linewidth}{
    PedralbesMonastery (2024) Heritage site \newline
    Forensic pathology analysis via numerical modeling of crack propagation, followed by the design of corrective measures for the stabilization of the historic 15th-century dome.}

\parbox{\linewidth}{
    Garellano residential tower (2025) Arup \newline
    Computational Fluid Dynamics (CFD) analysis to determine wind loads on the facade of a high-rise residential complex.}

\subsection{Computational Design \& BIM Automation (R\&D)}
\parbox{\linewidth}{
    Head of Development \& Innovation (2022-2025) SOCOTEC Spain - Buildings \& cities \newline
    Developed proprietary algorithms (C\# / .NET / Python) to enable direct interoperability between calculation models and BIM software (Revit/Robot), effectively eliminating manual rework and data entry errors.}

\parbox{\linewidth}{
    External consultant (2024 - 2025) Hormipresa \newline
    Implemented automated processes for the review of lifting assemblies in precast concrete elements, optimizing safety standards and significantly reducing technical review times.}


\printbibliography[heading={bibintoc}, title={Participation in R+D+i projects}, type=project]


\section{Education}
\begin{tabularx}{\linewidth}{lX}
    2017 - 2022 &
    \textbf{PhD in Civil Engineering} \newline
    UPC, Barcelona, Spain \\

    2013 - 2015 &
    \textbf{Bachelor's degree in Philosophy and Theology} \newline
    Pontificia Universtià della Santa Croce, Rome, Italy \\

    2007 - 2013 &
    \textbf{Ingeniero de Caminos, Canales y Puertos} (equivalent to MSc in Civil Eng.) \newline
    UPC, Barcelona, Spain \\
\end{tabularx}


\section{Selected publications}
\textit{Full list available on ORCID: \href{https://orcid.org/0000-0003-0962-8550}{0000-0003-0962-8550}}
\printbibliography[heading=none, keyword=highlight]
% \section{Publications}
% \printbibliography[heading={subbibliography}, title={Journal papers}, type=article]
% \printbibliography[heading={subbibliography}, title={Theses}, type=thesis]
% \printbibliography[heading={subbibliography}, title={Conferences}, type=inproceedings]


\section{Academic and professional activities}
\begin{itemize}
    \item \textbf{University teaching} (2017 - present): Adjunct Lecturer at UPC (Barcelona School of Civil Engineering). Subjects: Structural Dynamics (Master's Degree) and Numerical Methods (Bachelor's Degree)
    \item \textbf{International Research Stays}: Visiting Researcher at CIMAT (Mexico). Research on stabilized formulations for hyperbolic conservation laws (2017)
    \item \textbf{Industry Outreach}: Speaker at the Association of Structural Consultants (ACE) on "Structural Sustainability and Decarbonization Criteria" (2024, 2025)
\end{itemize}


% \section{Stays}
% \textbf{Visiting researcher, Centro de Investigación en Matemáticas.} Purpose: Research. Comparable tasks: WP3 of project
% TCAiNMaND-PIRSES-GA-2013-612607 Numerical Methods for Real Time Computations Research on stabilized
% formulations the Finite Element Method applied to hyperbolic conservation laws. 18/07/2017 - 19/10/2017.

% \section{Teaching activities}
% \begin{tabularx}{\linewidth}{lX}
%     2022 - \enspace \dots &
%     \textbf{Structural Dynamics.} Credits: 7'5. Degree: Master's degree in Structural \& Construction Engineering. UPC. \\

%     2022 - 2022 &
%     \textbf{Structural Analysis.} Credits: 7'5. Degree: Master's degree in Civil Engineering. UPC. \\

%     2017 - 2017 &
%     \textbf{Numerical Methods.} Credits: 6. Degree: Bachelor's degree in Materials Engineering. UPC. \\
% \end{tabularx}

% \section{Divulgation activities}
% \parbox{\linewidth}{
%     Subject: Sostenibilidad de las estructuras. Criterios para la descarbonización. \\
%     Promoting entity: Asociación de Consultores de Estructuras (ACE) \\
%     Organizing entity: Fundació Privada Institut d'Estudis Estructurals \\
%     Dates: 16/10/2024 – 13/11/2024; \hspace{2pt} 29/09/2025 – 03/11/2025}


\end{document}
