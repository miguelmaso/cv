\documentclass[11pt]{article}

\usepackage[spanish]{babel}
\usepackage{cv}

\addbibresource{cv.bib}

\author{Dr. Ing. Miguel Masó}

\begin{document}
\maketitle
\email{miguel.maso@socotec.com} \qquad +34 717 712 217 \\
Ciudad: Barcelona (08034) \qquad
Fecha de nacim.: 9/4/1989 \qquad
ORCID: 0000-0003-0962-8550


\section{Historial profesional}
\begin{tabularx}{\linewidth}{lX}
    2022 - actual &
    \textbf{SOCOTEC Spain} - Estructuras \newline
    Ingeniero Especialista, Responsable de innovación y desarrollo \\

    2022 - actual &
    \textbf{Universitat Politècnica de Catalunya (UPC)} - Escuela Técnica Superior de Ingeniería de Caminos, Canales y Puertos de Barcelona \newline
    Profesor asociado \\ 

    2017 - 2022 &
    \textbf{Centro Internacional de Métodos Numéricos en la Ingeniería (CIMNE)} \newline
    Estudiante de doctorado con beca FPI \\
\end{tabularx}


\section{Educación}
\begin{tabularx}{\linewidth}{lX}
    2017 - 2022 &
    \textbf{PhD in Civil Engineering} \newline
    UPC, Barcelona, España \\

    2013 - 2015 &
    \textbf{Bachelor's degree in Philosophy and Theology} \newline
    Ponitificia Universtià della Santa Croce, Roma, Italia \\

    2007 - 2013 &
    \textbf{Ingeniero de Caminos, Canales y Puertos} (Ingeniería superior) \newline
    UPC, Barcelona, Spain \\
\end{tabularx}


\section{Resumen}
Ingeniero de Caminos y Doctor en Mecánica Computacional con una trayectoria híbrida que combina la ingeniería estructural avanzada y el desarrollo de software.
Con experiencia en cerrar la brecha entre la teoría numérica y la aplicación industrial, con dominio de C++, Python, .NET y APIs de BIM.

Como Responsable de Desarrollo e Innovación en SOCOTEC, lideré la automatización de procesos de diseño y la interconectividad entre modelos de cálculo y BIM.
Mi formación doctoral en métodos numéricos (FEM, interacción fluido-estructura) me permite abordar problemas complejos de simulación física desde el código fuente hasta la solución práctica.
Actualmente investigador postdoctoral en CIMNE desarrollando formulaciones para nuevos materiales (polímeros electroactivos), busco aplicar mi capacidad técnica en entornos de desarrollo de software de ingeniería de alto nivel.


\section{Core competencies \& technical skills}
\begin{description}
    % \itemsep=-.3em
    \item[Advanced simulation \& physics]
        \emph{Computational mechanics}: Expertise in FEM, CFD, Particle-methods, Stabilization of hyperbolic equations.
        \emph{Structural analysis}: Vibration assessment, Disproportional collapse, CFD for Fire/Wind engineering, non-linear structural analysis.
        \emph{Ocean \& coastal}: Shallow water equations modelling, landslide-generated waves simulation.
    \item[Scientific programming \& automation]
        \emph{Languages}: C++ (high-performance solvers), Python \& Julia (Data processing/Scripting), C\# / .NET (Windows ecosystem).
        \emph{Plattforms}: Windows, Linux.
        \emph{BIM development}: Advanced interoperability using Revit API and Robot Structural Analysis API.
        \emph{Version control}: Git, CI/CD workflows.
    \item[Languages] Spanish (native) - Catalan (Native) - English (C1) - Italian (B1)
\end{description}


\section{Employment history}
\begin{tabularx}{\linewidth}{lX}
    2025 - \enspace \dots &
    \textbf{International Centre for Numerical Methods in Engineering (CIMNE)} \newline
    Postdoctoral researcher \\

    2022 - \enspace \dots &
    \textbf{Universitat Politècnica de Catalunya (UPC)} - School of Civil Engineering \newline
    Adjunct lecturer \\ 

    2022 - 2025 &
    \textbf{SOCOTEC Spain} - Buildings and cities \newline
    Specialist Engineer, Head of Development and Innovation \\

    2017 - 2022 &
    \textbf{International Centre for Numerical Methods in Engineering (CIMNE)} \newline
    PhD Candidate (FPI Fellowship) \\
\end{tabularx}


\section{Selected industry projects}

\subsection{Advanced Structural Analysis \& Complex Dynamics}
\parbox{\linewidth}{
    Mercury Tower, Malta (2024) Zaha Hadid \newline
    \emph{Reto}: Evaluación de dinámica estructural para una atracción de "teatro volador" (Flying Theatre) en un rascacielos de diseño singular. \newline
    \emph{Solución}: Realicé el análisis avanzado de vibraciones para garantizar el cumplimiento de estrictos límites de servicio (SLS) y confort humano bajo cargas dinámicas.}

\parbox{\linewidth}{
    New Velindre Cancer Centre, Cardiff (2022-2025) SACYR \newline
    \emph{Retos}: Diseño de infraestructura crítica sanitaria con requisitos estrictos de aislamiento y seguridad ante accidentes. \newline
    \emph{Soluciones}: Lideré los estudios de vibraciones para el aislamiento de equipos médicos de alta sensibilidad (MRI/Escáneres) asegurando la operatividad del centro y de robustez frente al colapso progresivo.}

\subsection{Multiphysics and forensic analysis}
\parbox{\linewidth}{
    Análisis Termo-mecánico (2023) SOCOTEC Francia \newline
    Ejecución de simulaciones acopladas CFD-Estructural para predecir modos de ruina en naves industriales bajo cargas de fuego, modelando deformaciones termo-plásticas.}

\parbox{\linewidth}{
    Monasterio de Pedralbes (2024) Patrimonio \newline
    Análisis forense de patologías mediante modelado numérico de la propagación de grietas y diseño de propuestas correctivas para la estabilización de la cúpula histórica (S. XV).}

\parbox{\linewidth}{
    Residencial Torre Garellano (2025) Arup \newline
    Análisis CFD para la determinación de cargas de viento en fachada en edificación de gran altura.}

\subsection{Computational Design \& BIM Automation (R\&D)}
\parbox{\linewidth}{
    Head of Development \& Innovation (2022-2025) SOCOTEC Spain - Buildings \& cities \newline
    Automatización: Desarrollo de algoritmos propios (C\# / .NET / Python) para la interoperabilidad directa entre modelos de cálculo y software BIM (Revit/Robot), eliminando el retrabajo manual.}

\parbox{\linewidth}{
    Consultor externo (2024 - 2025) Hormipresa \newline
    Control de Calidad: Implementación de procesos automatizados para la revisión de conjuntos de izado en prefabricados, optimizando la seguridad y reduciendo tiempos de revisión técnica.}



\section{Publicaciones}
\nocite{*}
\printbibliography[heading={subbibliography}, title={Artículos en revistas}, type=article]
\printbibliography[heading={subbibliography}, title={Tesis}, type=thesis]
\printbibliography[heading={subbibliography}, title={Presentaciones en congresos}, type=inproceedings]


\section{Estancias en centros extranjeros}
\textbf{Centro de Investigación en Matemáticas} Purpose: Research. Comparable tasks: WP3 of project
TCA\-iNMaND-PIRSES-GA-2013-612607 Numerical Methods for Real Time Computations Research on stabilized
formulations the Finite Element Method applied to hyperbolic conservation laws. 18/07/2017 - 19/10/2017.


\section{Actividad docente}
\begin{tabularx}{\linewidth}{lX}
    2022 - today &
    \textbf{Dinámica de estructuras.} 7'5 Créditos. Titulación: Máster en Ingeniería Estructural y de la Construcción. UPC. \\

    2022 - 2022 &
    \textbf{Análisis de estructuras.} 7'5 Créditos. Titulación: Máster en Ingeniería Civil. UPC. \\

    2017 - 2017 &
    \textbf{Métodos numéricos.} 6 Créditos. Titulación: Grado en Ingeniería de Materiales. UPC. \\
\end{tabularx}


\printbibliography[heading={bibintoc}, title={Participación en proyctos de I+D+i}, type=project]


\section{Divulgación}
\begin{tabularx}{\linewidth}{l}
    \parbox{\linewidth}{
        Curso: Sostenibilidad de las estructuras. Criterios para la descarbonización. \\
        Entidad promotora: Asociación de Consultores de Estructuras (ACE) \\
        Entidad organizadora: Fundació Privada Institut d'Estudis Estructurals \\
        Fechas: 16/10/2024 - 13/11/2024
    }
\end{tabularx}


\section{Otros}
\begin{itemize}
    \itemsep=-.3em
    \item \textbf{Palabras clave:} Mecánica computacional, Ingeniería de viento y fuego, Ingeniería marítima, Análisis de vibraciones
    \item \textbf{Programación:} C++, Python, C\#, .NET
    \item \textbf{Herramientas:} Robot Structural Analysis API, Revit API
    \item \textbf{Otros:} ENVISION SP 2023-2024
\end{itemize}


\end{document}
