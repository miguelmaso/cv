\documentclass[12pt]{article}

\usepackage{cv}
\usepackage[spanish]{babel}

\addbibresource{cv.bib}

\author{Dr. Ing. Miguel Masó}

\begin{document}
\maketitle
\email{miguel.maso@socotec.com} \qquad +34 717 712 217 \\
Ciudad: Barcelona (08006) \qquad
Fecha de nacim.: 9/4/1989 \qquad
ORCID: 0000-0003-0962-8550


\section{Historial profesional}
\begin{tabularx}{\linewidth}{lX}
    2022 - actual &
    \textbf{SOCOTEC Spain} - Estructuras \newline
    Specialist Engineer, Head of research and development \\

    2022 - actual &
    \textbf{Universitat Politècnica de Catalunya (UPC)} - Escuela Técnica Superior de Ingeniería de Caminos, Canales y Puertos de Barcelona \newline
    Profesor asociado \\ 

    2017 - 2022 &
    \textbf{Centro Internacional de Métodos Numéricos en la Ingeniería (CIMNE)} \newline
    Estudiante de doctorado con beca FPI \\
\end{tabularx}


\section{Educación}
\begin{tabularx}{\linewidth}{lX}
    2017 - 2022 &
    \textbf{PhD in Civil Engineering} \newline
    UPC, Barcelona, España \\

    2013 - 2015 &
    \textbf{Bachelor's degree in Philosophy and Theology} \newline
    Ponitificia Universtià della Santa Croce, Roma, Italia \\

    2007 - 2013 &
    \textbf{Ingeniero de Caminos, Canales y Puertos} (Ingeniería superior) \newline
    UPC, Barcelona, Spain \\
\end{tabularx}


\section{Idiomas}
Español nativo - Catalán nativo - C1 Inglés - B1 Italiano


\section{Resumen}
After studying civil engineering and a period studying philosophy and theology in Rome, he began
a doctorate in civil engineering at CIMNE. During the initial years of his doctoral studies, he combined
research with work for social purposes at the Montseny Foundation.

Durign the research period at CIMNE, he worked on the stabilization of finite elements for the hyperbolic
differential equations, with focus on numerical implementation, multi-physics coupling, and multi-scale
resolution. These developments have been succesfully applied to various projects. In addition to
the finite element stabilization, he also researched on the particle finite element method, either
with moving mesh and fixed mesh.

Currently, he is working at the Building and Cities department of Socotec-Spain. He has participated
in several international projects for structural design, taking on responsibilities of vibration
assessment and plasticity. He is leading the innovation and development department.


\section{Publicaciones}
\nocite{*}
\printbibliography[heading={subbibliography}, title={Artículos en revistas}, type=article]
\printbibliography[heading={subbibliography}, title={Tesis}, type=thesis]
\printbibliography[heading={subbibliography}, title={Presentaciones en congresos}, type=inproceedings]


\section{Estancias en centros extranjeros}
\textbf{Centro de Investigación en Matemáticas} Purpose: Research. Comparable tasks: WP3 of project
TCAiNMaND-PIRSES-GA-2013-612607 Numerical Methods for Real Time Computations Research on stabilized
formulations the Finite Element Method applied to hyperbolic conservation laws. 18/07/2017 - 19/10/2017.


\section{Actividad docente}
\begin{tabularx}{\linewidth}{lX}
    2022 - today &
    \textbf{Dinámica de estructuras.} 7'5 Créditos. Titulación: Máster en Ingeniería Estructural y de la Construcción. UPC. \\

    2022 - 2022 &
    \textbf{Análisis de estructuras.} 7'5 Créditos. Titulación: Máster en Ingeniería Civil. UPC. \\

    2017 - 2017 &
    \textbf{Métodos numéricos.} 6 Créditos. Titulación: Grado en Ingeniería de Materiales. UPC. \\
\end{tabularx}


\section{Participación en proyctos de I+D+i}
\begin{tabularx}{\linewidth}{l}
    \parbox{\linewidth}{
        Título: (2021 SGR 01349) Mecànica de Medis Continus i Estructures \\
        Entidad financiadora: AGAUR. Agència de Gestió d'Ajuts Universitaris i de Recerca \\
        Desde: 01/01/2022 Hasta: 30/06/2025 \\
        IP: Cervera Ruiz, Luis Miguel
    } \\ [2.5em]

    \parbox{\linewidth}{
        Título: (DPI2015-67857-R) Proyectos de I+D: Retos de la Sociedad 2015 \\
        Entidad financiadora: MINECO \\
        Desde: 01/01/2016 Hasta: 31/12/2018 \\
        IP: Codina, Ramon; Baiges, Joan
    } \\ [2.5em]

    \parbox{\linewidth}{
        Título: (FP7- 612607) FP7-PEOPLE-2013-IRSES \\
        Entidad financiadora: EC \\
        Desde: 01/01/2014 Hasta: 31/12/2017 \\
        IP: Larese de Tetto, Antonia
    }
\end{tabularx}


\section{Divulgación}
\begin{tabularx}{\linewidth}{l}
    \parbox{\linewidth}{
        Curso: Sostenibilidad de las estructuras. Criterios para la descarbonización. \\
        Entidad promotora: Asociación de Consultores de Estructuras (ACE) \\
        Entidad organizadora: Fundació Privada Institut d'Estudis Estructurals \\
        Fechas: 16/10/2024 - 13/11/2024
    }
\end{tabularx}


\section{Otros}
\begin{itemize}
    \itemsep=-.3em
    \item \textbf{Palabras clave:} Mecánica computacional, Ingeniería de viento y fuego, Ingeniería marítima, Análisis de vibraciones
    \item \textbf{Programación:} C++, Python, C\#, .NET
    \item \textbf{Herramientas:} Robot Structural Analysis API, Revit API
    \item \textbf{Otros:} ENVISION SP 2023-2024
\end{itemize}


\end{document}
